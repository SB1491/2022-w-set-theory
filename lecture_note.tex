\documentclass{article}
\usepackage[utf8]{inputenc}
\usepackage{amsmath}
\usepackage{amsfonts}
\usepackage{amsthm}
\usepackage{graphicx}
\usepackage{setspace}
\usepackage{caption}
\usepackage{subcaption}
\usepackage{listings}
\usepackage{forest}
\usepackage{amsfonts}
\usepackage{array, makecell}
\usepackage{exercise}
\usepackage{kotex}
\usepackage{amssymb}
\usepackage{algorithm}
\usepackage{titlesec}   
\usepackage{algpseudocode}
\usepackage{fancyhdr}
\title{Set Theory Lecture Note}
\author{TaeYoung Rhee}
\date{Winter 2022}
\theoremstyle{definition}
\newtheorem{defn}{Definition}[]
\newtheorem{thm}{Theorem}[]
\newtheorem{ex}{Example}[]
\newtheorem{pp}{Proposition}
\newtheorem{nt}{Notation}
\newtheorem{cor}{Corollary}
\newtheorem{lm}{Lemma}
\newtheorem*{rmk*}{Remark}
\newtheorem*{pf}{Proof}
\newenvironment{pf*}{\pushQED{\qed}\pf}{\popQED\endpf}
\setstretch{1.3}


\begin{document}

\maketitle

\section{ZFC}
Actually, there exists a 0th axiom.
\begin{align*}
    \exists x (x = x )
\end{align*}
It means there `is something'.
\begin{enumerate}
    \item Extensionality
    \begin{align*}
        \forall x \forall y ( \forall z ( z \in x \leftrightarrow z \in y)
        \rightarrow x = y) \\ 
        A \subseteq B , B \subseteq A \rightarrow A = b
    \end{align*}
    \item Foundation / Regularity
    \begin{align*}
        \forall x ( \exists y ( y \in x) \rightarrow \exists y ( y \in x \wedge \neg \exists z
        (z \in x \wedge z \in y)))
    \end{align*}
    \item Seperation Schema \\ 
    For each $\phi$ with free variables among $x, z, w_1, \cdots, w_n$
    \begin{align*}
        \forall w_1, \cdots, \forall w_n \forall x \exists y \forall z
        (z \in y \leftrightarrow z \in x \wedge \phi(x, z, w_1, \cdots, w_n) )
    \end{align*}
    \item Pairing
    \begin{align*}
        \forall x \forall y \exists z ( x \in z \wedge y \in z)
    \end{align*}
    \item Union
    \begin{align*}
        \forall x \exists y \forall z \forall w ((z \in w \wedge w \in x) \rightarrow z \in y)
    \end{align*}
    \begin{align*}
        \bigcup x = \cup_{w\in x} w
    \end{align*}
    \item Replacement scheme \\ 
    For each $\phi$ whose free variables among $x, z, w_1, \cdots, w_n$
    \begin{align*}
        \forall w_1 \cdots \forall w_n \forall x  \forall z(
            (z \in x \rightarrow \exists ! y \phi ) \rightarrow 
            \exists u (\forall z \exists y ( z \in x \rightarrow y \in u \wedge \phi))
        ) 
    \end{align*}
    \item Infinity
    \begin{align*}
        \exists x ( 0 \in x \wedge \forall y (y \in x \rightarrow S(y) \in x))
    \end{align*}
    \item Power set
    \begin{align*}
        \forall x \exists y \forall z  ( z \subseteq x \rightarrow z \in y)
    \end{align*}
    \item Choice 
    \begin{align*}
        \forall X \exists R (R \text{ well orders } X) 
    \end{align*}
    The following statments are equivalent to axiom of choice
    \begin{itemize}
        \item Every commutative ring with  $1$ has a maximal ideal.
        \item Every vector space over a field has a basis.
        \item Tychonoff's theorem
    \end{itemize}
\end{enumerate}
\end{document}